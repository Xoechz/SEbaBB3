\documentclass[12pt]{scrartcl}

\usepackage[utf8]{inputenc}
\usepackage[naustrian]{babel}
\usepackage{caption}
\usepackage{graphicx}
\usepackage{verbatim}
\usepackage[T1]{fontenc}
\usepackage{lmodern}
\usepackage{subcaption}
\usepackage{amsmath}
\usepackage{mathtools}
\usepackage{amsfonts}
\usepackage{listings}
\usepackage{csquotes}

%pdfs
\usepackage{pdfpages}
\usepackage{tikz}

%page borders
\usepackage{geometry}
\geometry{left=2.5cm,right=2.5cm,top=3cm,bottom=2.5cm}

% binary tree
\usepackage{tikz-qtree}

\newenvironment{code}{\captionsetup{type=listing}}{}

%title/footer/header values
\usepackage{titling}
\title{MAS3 Skript}
\author{SEbaBB 2024}
\date{\today{}, Hagenberg}

%footer/header
%\usepackage[automark]{scrpage2}
%\pagestyle{headings}
%\clearscrheadfoot
%\ihead{\thetitle}
%\chead{\theauthor}
%\ohead{\today}
%\cfoot{Seite \pagemark}

\begin{document}

\maketitle
\tableofcontents

\pagebreak

\section{Einführung}

Folgende Teilbereiche der Statistik werden behandelt:
\begin{itemize}
    \item Deskriptive Statistik
    \item Induktive Statistik
    \item Wahrscheinlichkeitsrechnung
\end{itemize}

\subsection{Übersicht Deskriptive und Induktive Statistik}

Die Deskriptive Statistik, auch beschreibende Statistik beschäftigt sich mit der Beschreibung von Daten.
Man versucht dabei Daten mihilfe von Lage- und Streuungsmaßen in einer verständlichen Form darzustellen.\par

Manchmal hat man keinen Zugriff zu allen Daten, dann bedient man sich der Induktiven Statistik, auch schließende Statistik. Mithilfe von Stichproben versuchtman Schlussfolgerungen auf die Grundgesamtheit zu ziehen.

Folgende Bereiche werden behandelt:
\begin{itemize}
    \item Lageparameter
    \item Streuungsparameter
    \item Lineare Transformationen
    \item Zusammenhangsmaße
\end{itemize}

\subsection{Übersicht Wahrscheinlichkeitsrechnung}

Die Wahrscheinlichkeitsrechnung beschäftigt sich mit der Quantifizierung von Zufäll. Zufälle sind Ereignisse, die nicht vollständig vorhersehbar sind.
Die Wahrscheinlichkeitsrechnung modeliert diese Zufälle und ermöglicht es, Aussagen über die Wahrscheinlichkeit von Ereignissen zu treffen.

Folgende Bereiche werden behandelt:
\begin{itemize}
    \item Laplace'sche Annahme
    \item Kombinatorik
    \item Axiome der Wahrscheinlichkeit
    \item Unaabhängigkeit
    \item Randwahrscheinlichkeit
    \item Diskrete und stetige Verteilungen
    \item Verteilungen
\end{itemize}

\section{Statistik}

\subsection{Grundbegiffe}

\begin{table}
    \begin{tabular}{ | c | c | }
        \hline
        Begriff                   & Definition                                              \\
        \hline
        \hline
        Statistische Masse        & Menge von statistischen Einheiten                       \\&\\
                                  & Die statistische Masse muss vor jeder statistischen     \\
                                  & Tätigkeit in räumlicher, zeitlicher und sachlicher      \\
                                  & Hinsicht präzise definiert werden.                      \\
        \hline
        Statistische Einheit      & Objekt, das Merkmale besitzt                            \\
        \hline
        Qualitatives Merkmal      & Merkmal, das nicht gemessen werden kann                 \\
                                  & (z.B. Geschlecht, Farbe)                                \\
        \hline
        Quantitatives Merkmal     & Merkmal, das gemessen werden kann und eine Einheit      \\
                                  & besitzt (z.B. Größe, Gewicht)                           \\
        \hline
        Primärstatistik           & Daten werden eigens für die statistische Zwecke erhoben \\
        \hline
        Sekundärstatistik         & Daten werden für andere Zwecke erhoben und für          \\
                                  & statistische Zwecke verwendet                           \\
        \hline
        Voll/Totalerhebung        & Alle statistischen Einheiten der Masse werden erfasst   \\
        \hline
        Teilerhebung/Stichprobe   & Nur ein Teil der s. Einheiten wird erfasst              \\
        \hline
        Zufallsstichprobe         & s. Einheiten werden zufällig gewählt                    \\
        \hline
        Repräsentative Stichprobe & s. Einheiten werden so gewählt, dass ihre Merkmale mit  \\
                                  & der gleicher relativen Häufigkeit (\emph{Quoten}), wie  \\
                                  & in der Gesamtmasse vorkommen.                           \\
        \hline

        \hline
    \end{tabular}
    \caption{Grundbegriffe}
\end{table}

\end{document}