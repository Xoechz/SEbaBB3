\documentclass[12pt]{scrartcl}

\usepackage[utf8]{inputenc}
\usepackage[naustrian]{babel}
\usepackage{caption}
\usepackage{graphicx}
\usepackage{verbatim}
\usepackage[T1]{fontenc}
\usepackage{lmodern}
\usepackage{subcaption}
\usepackage{amsmath}
\usepackage{mathtools}
\usepackage{amsfonts}
\usepackage{listings}
\usepackage{csquotes}

%pdfs
\usepackage{pdfpages}
\usepackage{tikz}

%page borders
\usepackage{geometry}
\geometry{left=2.5cm,right=2.5cm,top=3cm,bottom=2.5cm}

% binary tree
\usepackage{tikz-qtree}

\newenvironment{code}{\captionsetup{type=listing}}{}

%title/footer/header values
\usepackage{titling}
\title{MAS3 Skript}
\author{SEbaBB 2024}
\date{\today{}, Hagenberg}

%footer/header
%\usepackage[automark]{scrpage2}
%\pagestyle{headings}
%\clearscrheadfoot
%\ihead{\thetitle}
%\chead{\theauthor}
%\ohead{\today}
%\cfoot{Seite \pagemark}

\begin{document}

\maketitle
\tableofcontents

\pagebreak

\section{Einführung}

Folgende Teilbereiche der Statistik werden behandelt:
\begin{itemize}
    \item Deskriptive Statistik
    \item Induktive Statistik
    \item Wahrscheinlichkeitsrechnung
\end{itemize}

\subsection{Übersicht Deskriptive und Induktive Statistik}

Die Deskriptive Statistik, auch beschreibende Statistik beschäftigt sich mit der Beschreibung von Daten.
Man versucht dabei Daten mihilfe von Lage- und Streuungsmaßen in einer verständlichen Form darzustellen.\par

Manchmal hat man keinen Zugriff zu allen Daten, dann bedient man sich der Induktiven Statistik, auch schließende Statistik. Mithilfe von Stichproben versuchtman Schlussfolgerungen auf die Grundgesamtheit zu ziehen.

Folgende Bereiche werden behandelt:
\begin{itemize}
    \item Lageparameter
    \item Streuungsparameter
    \item Lineare Transformationen
    \item Zusammenhangsmaße
\end{itemize}

\subsection{Übersicht Wahrscheinlichkeitsrechnung}

Die Wahrscheinlichkeitsrechnung beschäftigt sich mit der Quantifizierung von Zufäll. Zufälle sind Ereignisse, die nicht vollständig vorhersehbar sind.
Die Wahrscheinlichkeitsrechnung modeliert diese Zufälle und ermöglicht es, Aussagen über die Wahrscheinlichkeit von Ereignissen zu treffen.

Folgende Bereiche werden behandelt:
\begin{itemize}
    \item Laplace'sche Annahme
    \item Kombinatorik
    \item Axiome der Wahrscheinlichkeit
    \item Unaabhängigkeit
    \item Randwahrscheinlichkeit
    \item Diskrete und stetige Verteilungen
    \item Verteilungen
\end{itemize}

\pagebreak

\section{Statistik}

\subsection{Grundbegiffe}

\begin{table}[h]
    \begin{tabular}{ | c | c | }
        \hline
        Begriff               & Definition                                                \\
        \hline
        \hline
        Statistische Masse    & Menge von statistischen Einheiten                         \\&\\
                              & Die statistische Masse muss vor jeder statistischen       \\
                              & Tätigkeit in räumlicher, zeitlicher und sachlicher        \\
                              & Hinsicht präzise definiert werden.                        \\
        \hline
        Statistische Einheit  & Objekt, das Merkmale besitzt.                             \\
        \hline
        Qualitatives Merkmal  & Merkmal, das nicht gemessen werden kann.                  \\
                              & (z.B. Geschlecht, Farbe)                                  \\
        \hline
        Quantitatives Merkmal & Merkmal, das gemessen werden kann und eine Einheit        \\
                              & besitzt. (z.B. Größe, Gewicht)                            \\
        \hline
        Primärstatistik       & Daten werden eigens für die statistische Zwecke erhoben.  \\
        \hline
        Sekundärstatistik     & Daten werden für andere Zwecke erhoben und für            \\
                              & statistische Zwecke verwendet.                            \\
        \hline
        Voll/Totalerhebung    & Alle statistischen Einheiten der Masse werden erfasst.    \\
        \hline
        Teilerhebung/         & Nur ein Teil der s. Einheiten wird erfasst.               \\
        Stichprobe            &                                                           \\
        \hline
        Zufallsstichprobe     & s. Einheiten werden zufällig gewählt.                     \\
        \hline
        Repräsentative        & s. Einheiten werden so gewählt, dass ihre Merkmale mit    \\
        Stichprobe            & der gleicher relativen Häufigkeit (\emph{Quoten}), wie    \\
                              & in der Gesamtmasse vorkommen.                             \\
        \hline
        Urliste               & Liste, die direkt bei der Datenerhebung entsteht.         \\
        \hline
        Stab/Balkendiagramm   & Diagramm, bei dem die Häufigkeit der Merkmalsausprägung   \\
                              & durch die Länge der Stäbe/Balken dargestellt wird.        \\
        \hline
        Klasse                & Intervall, in dem die Merkmalsausprägungen                \\
                              & zusammengefasst werden. (z.B. 0-10 Jahre, 11-20 Jahre...) \\
        \hline
        Kumulierte Häufigkeit & Summe der Häufigkeiten von einer Schranke bis zu einem    \\
                              & bestimmten Wert.                                          \\&\\
                              & Nur bei qualitativ ordinalen oder quantitativen Merkmalen \\
        \hline
    \end{tabular}
    \caption{Grundbegriffe 1}
\end{table}
\pagebreak
\begin{table}[h]
    \begin{tabular}{ | c | c | }
        \hline
        Begriff      & Definition                                                     \\
        \hline
        \hline
        Empirische   & Verteilung der Merkmalsausprägungen in einer Stichprobe.       \\
        Verteilung   &                                                                \\
                     & Gibt die kumulierte Häufigkeit der Merkmalsausprägungen        \\
                     & an. Die dafür Formel ist:                                      \\&\\
                     & $F(x) =\frac{Anzahl mit Merkmal \leq x}
        {Anzahl alle Elemente}$                                                       \\&\\
        \hline
        Lage         & Lageparameter helfen das Zentrum der Verteilung zu  bestimmen. \\
                     & (Modus, Median, Arithmetisches Mittel)                         \\
        \hline
        Streuung     & Streuungsparameter helfen zu ermitteln wie stark die Werte um  \\
                     & den Durchschnitt verteilt sind.                                \\
        \hline
        Schiefe      & Schiefe der Verteilung gibt an, ob die Verteilung symmetrisch  \\
                     & ist, oder ob sie nach links oder rechts neigt.                 \\
        \hline
        Wölbung      & Wölbung der Verteilung gibt an, ob die Verteilung spitz oder   \\
                     & flach ist, das heißt ob die Werte um den Durchschnitt geballt  \\
                     & sind oder ob sie gleichmäßig verteilt sind                     \\
        \hline
        Zusammenhang & Zusammenhangsmaße geben an, ob und wie stark zwei Merkmale     \\
                     & miteinander zusammenhängen.                                    \\
        \hline
    \end{tabular}
    \caption{Grundbegriffe 2}
\end{table}

\subsection{Notation}

\begin{table}[h]
    \begin{tabular}{ | c | c | }
        \hline
        Zeichen   & Bedeutung                               \\
        \hline
        \hline
        $N$       & Anzahl der Elemente der Grundgesamtheit \\
        $n$       & Anzahl der Elemente der Stichprobe      \\
        $\mu$     & Durchschnitt der Grundgesamtheit        \\
        $\bar{x}$ & Durchschnitt der Stichprobe             \\
        $\sigma$  & Standardabweichung der Grundgesamtheit  \\
        $s$       & Standardabweichung der Stichprobe       \\
        $q_n$     & n-tes Quantil                           \\
        \hline
    \end{tabular}
    \caption{Notation}
\end{table}

\subsection{Lageparameter}
\subsection{Streuungsparameter}
\subsection{Lineare Transformationen}
\subsection{Zusammenhangsmaße}
\section{Wahrscheinlichkeitsrechnung}
\subsection{Laplace'sche Annahme}
\subsection{Kombinatorik}
\subsection{Axiome der Wahrscheinlichkeit}
\subsection{Unabhängigkeit}
\subsection{Bedingte und Unbedingte Wahrscheinlichkeiten}

\pagebreak

\subsection{Diskrete und stetige Verteilungen}

Es gibt zwei Hauptarten von Verteilungen:
\begin{itemize}
    \item Diskrete Verteilungen
    \item Stetige Verteilungen
\end{itemize}

\subsubsection{Diskrete Verteilungen}

In einer diskreten Verteilung können nur bestimmte abzählbare Werte angenommen werden.
Die Werte sind oft ganzzahlig, zum Beispiel die Anzahl von Würfen bei einem Würfelspiel oder die Anzahl von Menschen in einer Schlange.\par

Jedes Ereignis hat eine Wahrscheinlichkeit zwischen 0 und 1.
Die Summe der Wahrscheinlichkeiten aller Ereignisse ist 1.\par

Die Wahrscheinlichkeitsfunktion $f(x)$ gibt die Wahrscheinlichkeit an, dass ein Ereignis $x$ eintritt.\par

\begin{align*}
    f(x) = P(X=x)
\end{align*}

Die Verteilungsfunktion $F(x)$ gibt die Wahrscheinlichkeit an, dass ein Ereignis kleiner oder gleich $x$ ist.
Sie wird durch die Summe der Wahrscheinlichkeiten aller Ereignisse bis $x$ berechnet.\par

\begin{align*}
    F(x) = P(X \leq x) = \sum_{i=1}^{x} f(x_i)
\end{align*}

Der Erwartungswert $\mu$ gibt den Durchschnitt der Verteilung an.\par

\begin{align*}
    \mu = E(x) = \sum_{i=1}^{n} x_i * f(x_i)
\end{align*}

Beispiele für diskrete Verteilungen sind die diskrete Gleichverteilung, die Hypergeometrische Verteilung, die Binomialverteilung und die Poissonverteilung.

\subsubsection{Stetige Verteilungen}

In einer stetigen Verteilung können unendlich viele Werte angenommen werden. Die Werte sind also nicht abzählbar.
Das heißt aber nicht das die Werte nicht begrenzt sind, sondern dass es unendlich viele Werte zwischen zwei Werten gibt.
Beispiele hierfür sind Körpergröße, Gewicht oder Temperatur.\par

Die Wahrscheinlichkeitsdichte $f(a,b)$ gibt die Wahrscheinlichkeit an, dass ein Ereignis in einem Intervall $[a,b]$ eintritt.\par

\begin{align*}
    f(x) = P(a \leq X \leq b)
\end{align*}

Die Wahrscheinlichkeit für ein genaues Ereignis ist 0, da es nur Wahrscheinlichkeiten für Intervalle gibt.\par

\begin{align*}
    P(X=x) = 0
\end{align*}

Die Verteilungsfunktion $F(x)$ gibt die Wahrscheinlichkeit an, dass ein Ereignis kleiner oder gleich $x$ ist.
Sie wird durch die Integration der Wahrscheinlichkeitsdichte von $-\infty$ bis $x$ berechnet.\par

\begin{align*}
    F(x) = P(X \leq x) = \int_{-\infty}^{x} f(x) dx
\end{align*}

Das Integral der Wahrscheinlichkeitsdichte über den gesamten Wertebereich ist 1.\par

\begin{align*}
    \int_{-\infty}^{\infty} f(x) dx = 1
\end{align*}

Der Erwartungswert $\mu$ gibt den Durchschnitt der Verteilung an.\par

\begin{align*}
    \mu = E(x) = \int_{-\infty}^{\infty} x * f(x) dx
\end{align*}

Die Varianz $\sigma^2$ gibt die Streuung der Verteilung an.
Sie gibt die Verteilung der Werte um den Erwartungswert an.
\par

\begin{align*}
    \sigma^2 = Var(x) = E(x^2) - E(x)^2
\end{align*}

Ein Beispiel für eine stetige Verteilung ist die Normalverteilung.

\subsection{Verschiebungssatz}

Der Verschiebungssatz besagt, dass der Erwartungswert einer linearen Transformation einer Zufallsvariablen gleich der linearen Transformation des Erwartungswerts ist.\par

\begin{align*}
    E(aX+b) = aE(X) + b
\end{align*}

\pagebreak

\subsection{Verteilungen}

Verteilungen sind mathematische Modelle, die die Wahrscheinlichkeit von Ereignissen beschreiben.

\subsubsection{Wie wähle ich die richtige Verteilung aus?}

\begin{table}[h]
    \begin{tabular}{ | c | c | }
        \hline
        Verteilung                   & Anwendungsbereich                             \\
        \hline
        \hline
                                     & Diskret                                       \\
        \hline
        Diskrete Gleichverteilung    & Zufällige Auswahl von Elementen               \\
        \hline
        Hypergeometrische Verteilung & Ziehen ohne Zurücklegen                       \\
        \hline
        Binomialverteiltung          & Ziehen mit Zurücklegen                        \\
        \hline
        Poissonverteilung            & Anzahl von Ereignissen in einem Zeitintervall \\
        \hline
        \hline
                                     & Stetig                                        \\
        \hline
        Normalverteilung             & Messwerte                                     \\
        \hline
    \end{tabular}
    \caption{Übersicht zu Verteilungen}
\end{table}

Für die Klausur unbedingt einen Taschenrechner mitnehmen der die Verteilungen berechnen kann,
da die Formeln sehr komplex sind.

\subsubsection{Diskrete Gleichverteilung}

Jedes Element hat die gleiche Wahrscheinlichkeit ausgewählt zu werden,
daher eignet sich die Gleichverteilung für die Zufällige Auswahl von Elementen.

\begin{table}[h]
    \begin{tabular}{ | c | c | }
        \hline
        Bezeichnung        & Formel                              \\
        \hline
        Untere Schranke    & $a$                                 \\
        Obere Schranke     & $b$                                 \\
        \hline             &                                     \\
        Wahrscheinlichkeit & $P(X=x) = \frac{1}{b-a+1}$          \\&\\&\\
        Erwartungswert     & $\mu = \frac{a+b}{2}$               \\&\\&\\
        Varianz            & $\sigma^2 = \frac{(b-a+1)^2-1}{12}$ \\&\\
        \hline
    \end{tabular}
    \caption{Diskrete Gleichverteilung}
\end{table}

\pagebreak

\subsubsection{Hypergeometrische Verteilung}

Die Hypergeometrische Verteilung eignet sich für das Ziehen ohne Zurücklegen.
Also wenn die Wahrscheinlichkeit für ein Ereignis nicht konstant ist und von vorherigen Ereignissen beeinflusst wird.

\begin{table}[h]
    \begin{tabular}{ | c | c | }
        \hline
        Bezeichnung                & Formel                                                           \\
        \hline
        Anzahl Elemente            & $N$                                                              \\
        Anzahl gewünschter Erfolge & $M$                                                              \\
        Anzahl Ziehungen           & $n$                                                              \\
        Anzahl gezogener Erfolge   & $m$                                                              \\
        \hline                     &                                                                  \\
        Wahrscheinlichkeit         & $P(X=x) = \frac{\binom{M}{m} * \binom{N-M}{n-m}}{\binom{N}{n}}$  \\&\\&\\
        Erwartungswert             & $\mu = \frac{n*M}{N}$                                            \\&\\&\\
        Varianz                    & $\sigma^2 = n * \frac{M}{N} * (1-\frac{M}{N}) * \frac{N-n}{N-1}$ \\&\\
        \hline
    \end{tabular}
    \caption{Hypergeometrische Verteilung}
\end{table}

\subsubsection{Binomialverteilung}

Die Binomialverteilung eignet sich für das Ziehen mit Zurücklegen.
Also wenn die Wahrscheinlichkeit für ein Ereignis konstant und unabhängig ist.

\begin{table}[h]
    \begin{tabular}{ | c | c | }
        \hline
        Bezeichnung        & Formel                                      \\
        \hline
        Anzahl Ziehungen   & $n$                                         \\
        Anzahl Erfolge     & $k$                                         \\
        Wahrscheinlichkeit & $p$                                         \\
        \hline             &                                             \\
        Wahrscheinlichkeit & $P(X=x) = \binom{n}{k} * p^k * (1-p)^{n-k}$ \\&\\&\\
        Erwartungswert     & $\mu = n * p$                               \\&\\&\\
        Varianz            & $\sigma^2 = n*p*(1-p)$                      \\&\\
        \hline
    \end{tabular}
    \caption{Binomialverteilung}
\end{table}

\subsubsection{Poissonverteilung}

Die Poissonverteilung eignet sich für die Anzahl von Ereignissen in einem Zeitintervall.
Die Ereignisse treten unabhängig voneinander auf und die Wahrscheinlichkeit für ein Ereignis ist konstant.

\begin{table}[h]
    \begin{tabular}{ | c | c | }
        \hline
        Bezeichnung        & Formel                                         \\
        \hline
        Anzahl Ereignisse  & $k$                                            \\
        Ereignisrate       & $\lambda$                                      \\

        \hline             &                                                \\
        Wahrscheinlichkeit & $P(X=k) = \frac{\lambda^k}{k!} * e^{-\lambda}$ \\&\\&\\
        Erwartungswert     & $\mu = \lambda$                                \\&\\&\\
        Varianz            & $\sigma^2 = \lambda$                           \\&\\
        \hline
    \end{tabular}
    \caption{Poissonverteilung}
\end{table}

\subsubsection{Normalverteilung}

Die Normalverteilung ist stetig und symmetrisch.
Sie beschreibt die kontinuierliche Verteilung von Messwerten.\par
Die Wahrscheinlichkeit für ein genaues Ereignis ist 0,
da es nur Wahrscheinlichkeiten für Intervalle gibt.

\begin{table}[h]
    \begin{tabular}{ | c | c | }
        \hline
        Bezeichnung        & Formel                                                                         \\
        \hline
        Erwartungswert     & $\mu$                                                                          \\
        Standardabweichung & $\sigma$                                                                       \\
        \hline             &                                                                                \\
        Wahrscheinlichkeit & $P(a \leq X \leq b) = \Phi(\frac{b-\mu}{\sigma}) - \Phi(\frac{a-\mu}{\sigma})$ \\&\\&\\
        Varianz            & $\sigma^2$                                                                     \\&\\
        \hline
    \end{tabular}
    \caption{Normalverteilung}
\end{table}

\end{document}